% Nejprve uvedeme tridu dokumentu s volbami
\documentclass[ing,male,java,dept460]{diploma}						% jednostranny dokument
%\documentclass[ing,male,java,dept460,twoside]{diploma}		% oboustranny dokument
\usepackage[czech]{babel}


% Zadame pozadovane vstupy pro generovani titulnich stran.
\ThesisAuthor{Ondřej Ždych}
% U bakalarske praxe neni nutne nazev zadavat
\ThesisTitle{Hlasový portál na platformě VoiceXML}

% U bakalarske praxe neni nutne anglicky nazev zadavat
\EnglishThesisTitle{Voice Portal with VoiceXML Platform}

\SubmissionDate{7. května 2014}

\PrintPublicationAgreement{false}

\Thanks{Touto cestou bych rád poděkoval vedoucímu diplomové práce, panu Ing. Pavlovi Nevludovi, především za konzultace a odborné rady, které jsem zúročil při psaní tohoto textu.}

\CzechAbstract{Tohle je nějaký abstrakt. Tohle je nějaký abstrakt. Tohle je nějaký abstrakt. Tohle je nějaký abstrakt.
Tohle je nějaký abstrakt. Tohle je nějaký abstrakt. Tohle je nějaký abstrakt. Tohle je nějaký abstrakt.
Tohle je nějaký abstrakt. Tohle je nějaký abstrakt. Tohle je nějaký abstrakt. Tohle je nějaký abstrakt.}

\CzechKeywords{typografie, \LaTeX, diplomová práce}

\EnglishAbstract{This is English abstract. This is English abstract. This is English abstract. This is English abstract. This is English abstract. This is English abstract.}

\EnglishKeywords{typography, \LaTeX, master thesis}

% Pridame pouzivane zkratky (pokud nejake pouzivame).
\AddAcronym{DVD}{Digital Versatile Disc}
\AddAcronym{TNT}{Trinitrotoluen}
\AddAcronym{OASIS}{Organization For The Advancement Of Structured Information Systems}
\AddAcronym{HTML}{Hyper Text Markup Language}


% Zadame cestu a jmeno souboru ci nekolika souboru s digitalizovanou podobou zadani prace
% Pri sazbe se pak hledaji soubory Figures/Zadani1.jpg, Figures/Zadani2.jpg atd.
% Do diplomove prace se postupne vlozi vsechny existujici soubory Figures/ZadaniXXX.jpg
% Pokud toto makro zapoznamkujeme sazi se stranka s upozornenim
\ThesisAssignmentImagePath{Figures/Zadani}

% Zadame soubor s digitalizovanou podobou prohlaseni
% Pokud toto makro zapoznamkujeme sazi se cisty text prohlaseni
%\DeclarationImageFile{Figures/Prohlaseni.jpg}


% Zacatek dokumentu
\begin{document}

% Nechame vysazet titulni strany.
\MakeTitlePages

% Asi urcite budeme potrebovat obsah prace.
\tableofcontents
\cleardoublepage	% odstrankujeme, u jednostranneho dokumentu o jednu stranku, u oboustrenneho o dve

% Jsou v praci tabulky? Pokud ano vysazime jejich seznam.
% Pokud ne smazeme nasledujici makro.
\listoftables
\cleardoublepage	% odstrankujeme, u jednostranneho dokumentu o jednu stranku, u oboustrenneho o dve

% Jsou v praci obrazky? Pokud ano vysazime jejich seznam.
\listoffigures
\cleardoublepage	% odstrankujeme, u jednostranneho dokumentu o jednu stranku, u oboustrenneho o dve


% Jsou v praci vypisy programu? Pokud ano vysazime jejich seznam.
\lstlistoflistings
\cleardoublepage	% odstrankujeme, u jednostranneho dokumentu o jednu stranku, u oboustrenneho o dve



% Zacneme uvodem
\section{Úvod}
\label{sec:Uvod}
Tady bude super úvod.

\section{Seznámení s problematikou VoiceXML}
\label{sec:Seznameni_s_vxml}
Něco o standardu VXML.

\subsection{Vývoj standardu}
Něco z historie.

\subsubsection{Verze 1.0}
Popis verze 1.

\subsubsection{Verze 2.0 a 2.1}
Popis verze 2.0 a 2.1.

\subsubsection{Budoucnost VXML}
Popis draftu 3.0.

\subsection{Základy VXML}
Základní info.

\subsubsection{Struktura VXML dokumentu}
Popis typické struktury VXML dokumentu. Členění aplikace do formu. Komunikace se serverem. (form, field)

\subsubsection{Uživatelský vstup}
Jak funguje zpracování vstupu od uřivatele. Popis gramatik a tagu, ktere se vztahuji ke sberu vstupu.

\subsubsection{Uživatelský výstup}
Popis možností generování výstupů pro uživatele. Tagy.

\subsection{Interpretry VXML}
Přehled vybraných interpretrů.

\subsubsection{Voxeo Prophecy}
Něco o Voxeu - licence, omezení, výhody (kvalitní TTS, podpora ASR), nevýhody (komerční).

\subsubsection{Asterisk + VoiceGlue}
Popis kombinace Asterisku a VoiceGlue - licence, omezení, výhody (opensource), nevýhody (nekvalitní TTS, nepodporuje ASR).

\section{Specifikace požadavků pro podporu automatického převodu textu do hlasu a obráceně (TTS a ASR)}
\label{sec:Podpora_TTS_a_ASR}
Obecný úvod o technologiích TTS a ASR.

\subsection{Technologie Text to Speech}
Popis procesu syntézy řeči.

\subsubsection{Flite}
Něco o Flite.

\subsubsection{Cepstral}
Něco o Cepstralu?

\subsubsection{Loquendo}
Něco o Loquendu.

\subsection{Technologie Automatic Speech Recognition}
Popis procesu převodu řeči na text.

\subsubsection{Loquendo}
Něco o ASR od Loquenda.

\section{Návrh webového (hlasového)? portálu s možnostmi převedení textu do hlasu a automatického rozpoznání hlasu}
\label{sec:Navrh}
Popis referenční aplikace. Seznam entit. Funkce aplikace: prihlaseni (admin, uzivatel), vyhledani letu, vytvoreni/zruseni rezervace, editace letu/dopravcu/uzivatelu. Propojeni web. a hlasoveho portalu pomoci WebSocket.

\subsection{Návrh webové portálu}
Popis použitých technologií (Angular.js, Node.js, Mongoose/MongoDB, Grunt, Jade, Javascript, WebSockety). Ukázka vybraných obrazovek aplikacce.

\subsection{Návrh hlasového portálu}
Popis použitých technologií. Nodejs.

\subsubsection{Návrh VXML knihovny}
Popis navrhu knihovny/architektury. Popis trid. Principy jak funguje (stavový automat, stavy, udalosti, přechody mezi stavy).

\paragraph{Srovnání návrhu aplikace pomocí knihovny a klasického přístupu}
Příklad zpracování vstupu a výstupu. Knihovna vx. VXML markup.
s
\subsubsection{Návrh VXML aplikace}
Popis funcionality hlasové aplikace (podmnožina funkcí webové aplikace). Popis jednotlivých částí aplikace. Komponenty. Ukázka diagramů částí aplikací.

\paragraph{Znovupoužitelné komponenty}
Zadání data (+diagram). Zadání textového vstupu (+diagram). Seznam rezervací/letů (+diagram).

\subsection{Konfigurace a zprovoznění hlasového portálu}
Popis, jak zprovoznit aplikaci na platformách. Voxeo Prophecy, Asterisk + VoiceGlue.

\subsubsection{Voxeo Prophecy}
...

\subsubsection{Asterisk + VoiceGlue}
...

\subsection{Doporučené postupy při návrhu}
Popis best practices. Jak spravne navrhnout dialog. Navrh struktury aplikace. Zamereni cilove skupiny uzivatelu.

\section{Zátěžové testy portálu}
\label{sec:Benchmark}
Popis, jak probíhal test. Použité nástroje.

\subsection{Benchmark aplikačního serveru}
ab test aplikačního serveru.

\subsection{Benchmark na platformě Voxeo Prophecy}
Omezení dva paralelní hovory. Test 1 a 2 připojených klientů.

\subsection{Benchmark na platformě VoiceGlue}
Test 1, 2, 5, 10 pčipojených klientů.

\subsection{Výsledky měření}
Co jsme zjistili v testech?

\section{Závěr}
\label{sec:Conclusion}
Tak tady je konečně konec.
\cite{goossens94,lamport94}.

\bigskip
\begin{flushright}
Ondřej Ždych
\end{flushright}

\begin{thebibliography}{99}

\bibitem{goossens94} Goossens, Michel,
\textit{The \LaTeX\ companion,} New York: Addison, 1994.

\bibitem{lamport94} Lamport, Leslie,
\textit{\LaTeX: a document preparation system: user's guide and reference manual},
New York: Addison-Wesley Pub. Co., 1994.

\end{thebibliography}


\appendix
\section{Grafy a měření}
Tohle je příloha k práci. Většinou se sem dávají grafy, tabulky, které by vzhledem
ke svému počtu překážely v textu diplomky.
\clearpage

\InsertFigure{Figures/Graf}{0.7\textwidth}{Nějaký graf}{fig:SampleGraph}

\end{document}
